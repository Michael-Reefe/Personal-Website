%%%%%%%%%%%%%%%%%%%%%%%%%%%%%%%%%%%%%%%%%%%%%%%%%%%%%%%%%%%%%%%%%%%%%%%%%%%%%%%%
% Medium Length Graduate Curriculum Vitae
% LaTeX Template
% Version 1.2 (3/28/15)
%
% This template has been downloaded from:
% http://www.LaTeXTemplates.com
%
% Original author:
% Rensselaer Polytechnic Institute 
% (http://www.rpi.edu/dept/arc/training/latex/resumes/)
%
% Modified by:
% Daniel L Marks <xleafr@gmail.com> 3/28/2015
%
% Important note:
% This template requires the res.cls file to be in the same directory as the
% .tex file. The res.cls file provides the resume style used for structuring the
% document.
%
%%%%%%%%%%%%%%%%%%%%%%%%%%%%%%%%%%%%%%%%%%%%%%%%%%%%%%%%%%%%%%%%%%%%%%%%%%%%%%%%

%-------------------------------------------------------------------------------
%	PACKAGES AND OTHER DOCUMENT CONFIGURATIONS
%-------------------------------------------------------------------------------

%%%%%%%%%%%%%%%%%%%%%%%%%%%%%%%%%%%%%%%%%%%%%%%%%%%%%%%%%%%%%%%%%%%%%%%%%%%%%%%%
% You can have multiple style options the legal options ones are:
%
%   centered:	the name and address are centered at the top of the page 
%				(default)
%
%   line:		the name is the left with a horizontal line then the address to
%				the right
%
%   overlapped:	the section titles overlap the body text (default)
%
%   margin:		the section titles are to the left of the body text
%		
%   11pt:		use 11 point fonts instead of 10 point fonts
%
%   12pt:		use 12 point fonts instead of 10 point fonts
%
%%%%%%%%%%%%%%%%%%%%%%%%%%%%%%%%%%%%%%%%%%%%%%%%%%%%%%%%%%%%%%%%%%%%%%%%%%%%%%%%

\documentclass[marg, centered]{res}

% Default font is the helvetica postscript font
\usepackage{charter}
\usepackage{xcolor}
\usepackage{inputenc}
\usepackage[hidelinks]{hyperref}
\usepackage{csquotes}
\usepackage{orcidlink}
\usepackage{fontawesome}
\usepackage{multibib}
\usepackage[numbers]{natbib}
\usepackage[english]{babel}
\usepackage{etaremune,etoolbox}
\usepackage{fancyhdr}
\usepackage{lastpage}
\usepackage{geometry}
 \geometry{
 letterpaper,
 total={8.5in,11in},
 left=20mm,
 top=20mm,
 right=53.5mm,
 bottom=10mm,
 footskip=0mm
 }

\pagestyle{fancy}
\renewcommand{\headrulewidth}{0pt}
\fancyhf{}
\fancyhfoffset[l]{\dimexpr-20mm+53.5mm\relax}
\cfoot{Page \thepage \hspace{1pt} of \pageref{LastPage}}

\urlstyle{same}
\definecolor{dkbu}{HTML}{180f7d}

% define helper functions
\newcommand{\CVItem}[1]{
  \item\small{
    {#1 \vspace{-2pt}}
  }
}
\newcommand{\aap}{A\&A}
\newcommand{\mnras}{MNRAS}
\newcommand{\apjs}{ApJS}
\newcommand{\aj}{AJ}
\newcommand{\apj}{ApJ}
\newcommand{\pasp}{PASP}
\newcommand{\jatis}{JATIS}

% Increase text height
\textheight=700pt

\bibliographystyle{msacs}
\newcites{first}{{\bfseries First \& Second Author}\vspace{0.1cm}}
\bibliographystylefirst{msacs}
\newcites{co}{{\bfseries Coauthor}\vspace{0.1cm}}
\bibliographystyleco{msacs}

%%%------------------------------------------------------------------
%%% Reverse numbering of bibliography
\makeatletter
\AtBeginDocument{%%% natbib redefines the environment there
\renewenvironment{thebibliography}[1]
 {\bibsection\parindent\z@\bibpreamble\bibfont
  \settowidth{\dimen0}{#1.}%
  \setlength{\dimen2}{\dimen0}%
  \addtolength{\dimen2}{\labelsep}
  \begin{etaremune}[labelwidth=\dimen0,leftmargin=\dimen2]
  \ifNAT@openbib
    \renewcommand\newblock{\par}%
  \else
    \renewcommand\newblock{\hskip.11em \@plus .33em \@minus .07em}%
  \fi
  \sloppy
  \clubpenalty4000
  \widowpenalty4000
  \sfcode`\.\@m
  \let\NAT@bibitem@first@sw\@firstoftwo
  \let\citeN\cite\let\shortcite\cite\let\citeasnoun\cite}
 {\bibitem@fin\bibpostamble
  \def\@noitemerr{\PackageWarning{natbib}{Empty `thebibliography' environment}}%
  \end{etaremune}
  \bibcleanup}
}%%% end of \AtBeginDocument

%%% patch \@lbibitem to use only \item (for etaremune)
\patchcmd{\@lbibitem}{\item[\hfil\NAT@anchor{#2}{\NAT@num}]}{\item}{}{}
\makeatother
% Remove natbib titles that cause problems with asterisks on the left margin
\renewcommand{\bibsection}{}

\definecolor{orcidlogocol}{HTML}{A6CE39}

\begin{document}

%-------------------------------------------------------------------------------
%	NAME AND ADDRESS SECTION
%-------------------------------------------------------------------------------
\name{Michael A. Reefe}

% Note that addresses can be used for other contact information:
% -phone numbers
% -email addresses
% -linked-in profile

\address{  MIT Kavli Institute, 37-602
\\70 Vassar St
\\Cambridge, MA 02139-4307}
% \address{George Mason University
% \\4400 University Dr
% \\Fairfax, VA 22030}
\address{ \faMobile \hspace{0.2cm} +1 (540) 848-4434
\\ \faEnvelope \hspace{0.1cm} \href{mailto:mreefe@mit.edu}{\color{dkbu} mreefe@mit.edu}
\\ \faGlobe \hspace{0.15cm} \href{https://www.mit.edu/~mreefe/}{\color{dkbu}www.mit.edu/$\sim$mreefe/}}
\address{ \faGithub \hspace{0.1cm} \href{https://github.com/Michael-Reefe}{\color{dkbu} Michael-Reefe}
\\ \faLinkedinSquare \hspace{0.1cm} \href{https://www.linkedin.com/in/michael-reefe/}{\color{dkbu} Michael Reefe}
\\ \orcidlink{0000-0003-4701-8497} \hspace{0.1cm} \href{https://orcid.org/0000-0003-4701-8497}{\color{dkbu} 0000-0003-4701-8497}
}

\begin{resume}

%-------------------------------------------------------------------------------
%	EDUCATION SECTION
%-------------------------------------------------------------------------------
\section{{\scshape \bfseries Education}}
\textbf{Ph.D., Physics} \hfill {\bf 2022 -- Present} \\
Massachusetts Institute of Technology, Cambridge, MA \\
{\em Advisor: Michael McDonald} \\
\vspace{-0.7cm}

\textbf{B.S., Physics $\mathbf{|}$ Concentration in Astrophysics} \hfill {\bf 2018 -- 2022}\\
With Honors, \textit{Summa cum laude} \hfill GPA: 4.00 \\
George Mason University, Fairfax, VA
% \textbf{Advanced Studies Diploma} \hfill {\bf 2014 -- 2018}\\
% Fort Defiance High School \hfill GPA: 4.46\\
% Shenandoah Valley Governor's School
%-------------------------------------------------------------------------------

%-------------------------------------------------------------------------------
%	EXPERIENCE SECTION
%-------------------------------------------------------------------------------
% Modify the format of each position
\begin{format}
\title{l}\dates{r}\\
\employer{l}\location{r}\\
\body
\end{format}
%-------------------------------------------------------------------------------

\section{{\scshape \bfseries Research Experience}}
\employer{Massachusetts Institute of Technology}
\location{Cambridge, MA}
\dates{\textbf{Aug. 2022 -- Present}}
\title{\textbf{Graduate Research Fellow}}
\begin{position}
\vspace{-0.35cm}
\begin{itemize}
    \CVItem{Goals: Studying galaxy clusters, the intracluster medium, and AGN feedback in the most extreme cool-core environments to understand the structure and evolution of the central galaxy and the surrounding ICM, and the cycle of feeding and feedback.}
    \CVItem{Observational Techniques: Multiwavelength integral field spectroscopy and photometry.}
    \CVItem{Data Analysis Techniques: Scientific coding in Python and Julia (Jupyter). Parallelization on computing clusters.}
\end{itemize}
\end{position}

\employer{George Mason University}
\location{Fairfax, VA}
\dates{\textbf{Sept. 2019 -- July 2022}}
\title{\textbf{Undergraduate Research Assistant}}
\begin{position}
\vspace{-0.35cm}
\begin{itemize}
    \CVItem{Goals: (1) Analyzing and modeling photometric transits and spectroscopic radial velocity data to validate exoplanets and model for planet characteristics. (2) Analyzing a large survey of nearly a million galaxy spectra from the Sloan Digital Sky Survey to search for heavily obscured or dim active galactic nuclei via coronal line emission.}
    \CVItem{Observational Techniques: (1) time-domain photometric and spectroscopic observations, and (2) multiwavelength integrated field unit spectroscopy.} 
    \CVItem{Data Analysis Techniques: Python coding, including complete hardware automation of the GMU campus telescope, scientific computing, AI/machine learning. Computing cluster integration and usage.}
\end{itemize}
\end{position}

\section{{\scshape \bfseries Teaching Experience}}
\employer{George Mason University}
\location{Fairfax, VA}
\dates{\textbf{June 2020 -- May 2020}}
\title{\textbf{Learning Assistant}}
\begin{position}
\vspace{-0.35cm}
\begin{itemize}
    \CVItem{Introductory electricity \& magnetism course}
    \CVItem{Responsibilities: Attending classes and answering students’ questions, helping them with problems. Holding personal office hours to work through examples. Creating a presentation to summarize the lessons learned from participating in this position (and the unique challenges faced by the COVID-19 pandemic).}
\end{itemize}
\end{position}
%-------------------------------------------------------------------------------

\begin{format}
\title{l} \\
\body
\end{format}


\section{{\scshape \bfseries Honors \&\\ Awards}}

\title{\textbf{NSF Graduate Research Fellowship} \hfill \textbf{2022 -- 2027}}
\begin{position}
\$37,000 stipend \& \$16,000 educational allowance per year for 3 years. Competitive national research fellowship for prospective graduate students across all science \& math disciplines that requires a detailed 3-year research proposal plan.
\end{position}

\vspace{-0.3cm}
\title{\textbf{MIT Whiteman Fellowship} \hfill \textbf{2022 -- 2023}}
\begin{position}
MIT Physics Department fellowship covering the full stipend and tuition for the first year of study, funded by the Patrons of Physics Fellows at MIT.
\end{position}

\vspace{-0.3cm}
\title{\textbf{Dean's Award for Excellence in Academics and Research} \hfill {\textbf{2022}}}
\begin{position}
\$1,250 award.  GMU College of Science award for excellence in academics and/or research.
\end{position}

\vspace{-0.3cm}
\title{\textbf{Outstanding Undergraduate Research Award} \hfill \textbf{2022}}
\begin{position}
GMU Physics \& Astronomy department recognition of exceptional undergraduate research.
\end{position}

\vspace{-0.3cm}
\title{\textbf{Outstanding Graduating Senior Award} \hfill \textbf{2022}}
\begin{position}
GMU Physics \& Astronomy department recognition of an exceptional graduating senior.
\end{position}

\vspace{-0.3cm}
\title{\textbf{Outstanding Learning Assistant Award} \hfill \textbf{2021}}
\begin{position}
\$150 award.  Recognition of outstanding leadership as a learning assistant.
\end{position}

\vspace{-0.3cm}
\title{\textbf{Osher Lifelong Learning Institute Scholarship} \hfill \textbf{2020}}
\begin{position}
\$500 award.  Recognition of academic excellence for GMU students.
\end{position}

\vspace{-0.3cm}
\title{\textbf{George Mason University Distinction Scholarship} \hfill \textbf{2018 -- 2022}}
\begin{position}
\$2,000 per year. Merit based scholarship for academically distinguished GMU students.
\end{position}

\vspace{-0.3cm}
\title{\textbf{Dean's List} \hfill \textbf{2018 -- 2022}}
\begin{position}
Cumulative GPA above 3.5 at GMU.
\end{position}

%-------------------------------------------------------------------------------
%	PROJECTS SECTION
%-------------------------------------------------------------------------------
% \newpage
\section{{\scshape \bfseries Refereed Publications}}

\vspace{-0.1cm}
\textbf{First Author}
\vspace{0.05cm}
\nocitefirst{*}
\bibliographyfirst{first}

\textbf{Coauthor}
\vspace{0.05cm}
\nociteco{*}
\bibliographyco{co}
\vspace{0.05cm}


\begin{format}
\title{l} \\
\employer{l} \\
\body
\end{format}

\section{{\scshape \bfseries Conferences \& Presentations}}

\begin{etaremune}

\item \textbf{25 Years of Science with Chandra \hfill \textbf{2--6 Dec. 2024}}\\
\textit{(Talk)} \hfill \textit{Boston, MA}\\
Mapping the Cooling Flow in the Phoenix Cluster with JWST and Chandra
\href{https://cxc.cfa.harvard.edu/cdo/symposium_2024/schedule.html#talk}{\texttt{[Website]}}

\item \textbf{243\textsuperscript{rd} Meeting of the American Astronomical Society \hfill \textbf{7--11 Jan. 2024}}\\
\textit{(Talk)} \hfill \textit{New Orleans, LA}\\
Shaken or stirred? Dynamics of the coronal temperature gas in the Phoenix Cluster
\href{https://ui.adsabs.harvard.edu/abs/2024AAS...24344204R/abstract}{\texttt{[ADS]}}
    
\item \textbf{240\textsuperscript{th} Meeting of the American Astronomical Society \hfill \textbf{12--17 June 2022}}\\
\textit{(iPoster)} \hfill \textit{Pasadena, CA}\\
CLASS: Coronal Line Activity in the Sloan Digital Sky Survey \href{https://ui.adsabs.harvard.edu/abs/2022AAS...24010113R/abstract}{\texttt{[ADS]}}

\item \textbf{TESS Science Conference II} \hfill \textbf{2--6 Aug. 2021}\\
\textit{(Poster)} \hfill \textit{Virtual}\\
A Flexible Python Observatory Automation Framework for the George Mason\\ University Campus Telescope \href{https://zenodo.org/record/5115310}{\texttt{[zenodo]}}

\item \textbf{GMU College of Science Undergraduate Research Colloquium} \hfill \textbf{22 Apr. 2021}\\
\textit{(Poster)} \hfill \textit{Virtual}\\
Automation of TESS Follow-up Observations with the GMU Campus Telescope

\item \textbf{237\textsuperscript{th} Meeting of the American Astronomical Society} \hfill \textbf{11--15 Jan. 2021}\\
\textit{(iPoster)} \hfill \textit{Virtual}\\
An Asynchronous Object-Oriented Approach to Automation of the 0.8-meter\\ George Mason University Campus Telescope in Python \href{https://ui.adsabs.harvard.edu/abs/2021AAS...23734407R/abstract}{\texttt{[ADS]}}

% \textit{Poster (Contributer)}\\
% Methods of Data Analysis on TESS Observations \href{https://zenodo.org/record/5115310}{\texttt{[zenodo]}} \\
% \textit{Poster (Contributer)}\\
% Transit Timing Variations for AU Microscopii b \& c \href{https://zenodo.org/record/5114040}{\texttt{[zenodo]}}


\end{etaremune}


\section{{\scshape \bfseries Proposals}}
\textbf{Co-I} \\
HST/COS Cycle 32 (24 orbits) \hfill 2024 \\
\textit{Probing Multiphase Cooling Via OVI Emission in the Cores of the Most Extreme Cooling Flows}

% Gemini Observatory: MAROON-X Instrument, 24$+$ nights requested \hfill 2021B \\
% Keck Observatory: HIRES Instrument, 5 nights requested \hfill 2021B \\
% NASA IRTF: iSHELL Instrument, 50 nights requested \hfill 2021B \\
% Gemini Observatory: MAROON-X Instrument, 24$+$ nights requested \hfill 2022A \\

\begin{format}
\title{l}\dates{r} \\
\employer{l} \\
\body
\end{format}

\section{{\scshape \bfseries Community Involvement \& Outreach:\\ Held Positions}}

\title{\textbf{PGSC Vice President of Academic Advocacy}}
\dates{\textbf{July 2024 -- Present}}
\employer{\textit{MIT Physics Graduate Student Council (PGSC)} \hfill Cambridge, MA}
\begin{position}
Served as the MIT PGSC's primary advocate for the students to the physics department leadership, holding regular meetings on how aspects of the PhD program and the department can be improved, i.e. more explicitly defined guidelines for academic advisors and academic advising meetings, uniform sets of expectations for the oral qualifying exam across the different physics divisions, implementing professional development requirements, etc. [\href{https://physics-gsc.scripts.mit.edu/home/}{\color{dkbu} PGSC Homepage}]
\end{position}

\title{\textbf{MIT Admissions Advisory Council Member}}
\dates{\textbf{July 2024 -- Present}}
\employer{\textit{MIT Physics Graduates Advising Graduate Admissions (GAGA)} \hfill Cambridge, MA}
\begin{position}
Served on the Admissions Advisory Council, AKA GAGA, which is a subcommittee of the Physics Graduate Student Council that advises the MIT Chair of Graduate Admissions from the graduate student perspective and organizes the PhysGAAP program (see below). [\href{https://physics-gsc.scripts.mit.edu/home/gaga/}{\color{dkbu} GAGA Page}]
\end{position}

\title{\textbf{PGSC Webmaster}}
\dates{\textbf{July 2023 -- Present}}
\employer{\textit{MIT Physics Graduate Student Council (PGSC)} \hfill Cambridge, MA}
\begin{position}
Served as the MIT PGSC's webmaster, maintaining the website, mailing lists, and calendar, and keeping them all up-to-date.
\end{position}

\title{\textbf{Astrogazers Member}}
\dates{\textbf{September 2023 -- Present}}
\employer{\textit{MIT Astrogazers} \hfill Cambridge, MA}
\begin{position}
As a member of the Astrogazers, I have been involved in engaging with the public at a number of sidewalk observing nights, exhibits at the annual Cambridge Science Festival, and other miscellaneous science-themed events in the greater Boston/Cambridge area. [\href{https://astrogazers.mit.edu/}{\color{dkbu} Astrogazers Homepage}]
\end{position}

\title{\textbf{MIT PhysGAAP Mentor}}
\dates{\textbf{December 2024}}
\employer{\textit{MIT Physics Graduate Application Assistance Program (PhysGAAP)} \hfill Cambridge, MA}
\begin{position}
Served as a mentor in MIT's PhysGAAP Program, aiding prospective PhD students (primarily from under-represented groups) with the MIT Physics application, providing guidance on how to navigate the application and how best to present themselves.  I have mentored 3 prospective students through this program. [\href{https://sites.mit.edu/physgaap/}{\color{dkbu} PhysGAAP Website}]
\end{position}

\title{\textbf{MKI Graduate Student Lunch Organizer}}
\dates{\textbf{July 2023 -- July 2024}}
\employer{\textit{MIT Kavli Institute for Astrophysics and Space Research (MKI)} \hfill Cambridge, MA}
\begin{position}
Organized weekly lunch catering and a talk series for the graduate students in MKI.
\end{position}

\title{\textbf{Faculty Search Undergraduate Liaison}}
\dates{\textbf{Jan. 2022}}
\employer{\textit{GMU Department of Physics \& Astronomy} \hfill Fairfax, VA}
\begin{position}
Worked as the undergraduate representative during a faculty search for a new astrophysics professor at GMU. Attended a mock lecture and research colloquium presented by each candidate, as well as interviews, and provided feedback to the faculty hiring committee from the undergraduate student perspective.
\end{position}

\title{\textbf{Spectrum President}}
\dates{\textbf{July 2021 -- Aug. 2022}}
\employer{\textit{Spectrum} \hfill Fairfax, VA}
\begin{position}
Planning talks, discussions, fundraisers, and other events, as well as managing website and budgetary concerns and working with the College of Science Faculty to improve diversity at GMU for student-led group \href{https://gmuspectrum.squarespace.com/}{\color{dkbu}{Spectrum}}, which promotes the enhancement of under-represented groups in STEM.
\end{position}

\title{\textbf{Spectrum Peer Mentor}}
\dates{\textbf{Dec. 2020 -- Aug. 2022}}
\employer{\textit{Spectrum} \hfill Fairfax, VA}
\begin{position}
Providing academic and personal tutoring for students in physics and astronomy at GMU through \href{https://gmuspectrum.squarespace.com/}{\color{dkbu}{Spectrum}}.
\end{position}

\title{\textbf{ASSIP Research Mentor}}
\dates{\textbf{Summer 2020, 2021}}
\employer{\textit{Aspiring Scientists' Summer Internship Program (ASSIP)} \hfill Fairfax, VA}
\begin{position}
Taught high school interns about the academic research done in our group, and tutored them on how to perform it themselves to synthesize a presentable project by the end of the summer.
\end{position}


\section{{\scshape \bfseries Community Involvement \& Outreach:\\ Events}}

\title{\textbf{MIT Museum After Dark: Trivia Cohost}}
\dates{\textbf{December 2024}}
\employer{\textit{MIT Astrogazers} \hfill Cambridge, MA}
\begin{position}
Worked with the Astrogazers to cohost trivia on the \textit{Hubble Space Telescope} at one of the MIT Museum's After Dark events themed around the 90s.
\end{position}

\title{\textbf{Cambridge Science Festival: Astrogazers}}
\dates{\textbf{September 2023, 2024}}
\employer{\textit{MIT Astrogazers} \hfill Cambridge, MA}
\begin{position}
Worked at the Astrogazers booths for the Cambridge Science Festival.
\end{position}

\title{\textbf{Sidewalk Observing Nights: Observer}}
\dates{\textbf{July 2023, May 2024}}
\employer{\textit{MIT Astrogazers} \hfill Cambridge, MA}
\begin{position}
Worked with the Astrogazers to set up telescopes for nighttime observing and talk to passersby about outer space and science in general.
\end{position}

\title{\textbf{MIT Physics Orientation: Organizer}}
\dates{\textbf{September 2023, 2024}}
\employer{\textit{MIT Physics Graduate Student Council} \hfill Cambridge, MA}
\begin{position}
Worked with the PGSC to organize events for the MIT Physics Department's annual Orientation, where students who have accepted their offer to join the physics PhD program get acquainted with the department.  PGSC organizes primarily social events for students to get to know their cohort.
\end{position}


\title{\textbf{MIT Physics Open House: Organizer}}
\dates{\textbf{April 2023, 2024}}
\employer{\textit{MIT Physics Graduate Student Council} \hfill Cambridge, MA}
\begin{position}
Worked with the PGSC to organize events for the MIT Physics Department's annual Open House, where prospective students who have been accepted to the physics PhD program come visit the department before deciding whether or not to accept their offer.  Events include a tour of the graduate student housing, liquid nitrogen ice cream social, panels with current students and faculty, etc.
\end{position}


\title{\textbf{College of Science Graduation: Speaker}}
\dates{\textbf{May 2022}}
\employer{\textit{GMU College of Science} \hfill Fairfax, VA}
\begin{position}
Chosen to be the student speaker for the College of Science's Spring 2022 graduation event. [\href{https://www.youtube.com/watch?v=xsyi9sqYH4o}{\color{dkbu} Recording}].
\end{position}

\title{\textbf{ASSIP Career Day: Panelist}}
\dates{\textbf{Aug. 2022}}
\employer{\textit{Aspiring Scientists' Summer Internship Program (ASSIP)} \hfill Fairfax, VA}
\begin{position}
Served on a panel of graduate students for a Career Day event hosted by GMU's ASSIP program, answering high school students' questions about a career in academia.
\end{position}

\title{\textbf{NSF GRFP Cohort Workshop: Panelist}}
\dates{\textbf{July 2022}}
\employer{\textit{GMU Office of Fellowships} \hfill Fairfax, VA}
\begin{position}
Served on a panel of NSF GRFP recipients and reviewers to answer students' questions about the application and review process.
\end{position}


%-------------------------------------------------------------------------------
%	COMPUTER SKILLS SECTION
%-------------------------------------------------------------------------------
\section{COMPUTER\\SKILLS}

\textbf{Coding}{: Python (Numpy, Numba, Scipy, Astropy, Pandas, Matplotlib, Plotly),
Julia, MATLAB, Mathematica, Bash, Git} \\ 
\textbf{Astronomy Programs}{: DS9, AstroImageJ} \\
\textbf{Document Creation}{: \LaTeX, Vim, Microsoft Office} \\
%-------------------------------------------------------------------------------


%	Interests
%-------------------------------------------------------------------------------
% \newpage
\section{{\scshape \bfseries References}}
\textbf{Michael McDonald}{: MIT, Associate Professor, PhD research advisor.}\\
\textbf{Shobita Satyapal}{: GMU, Professor, Undergraduate research advisor.}\\
\textbf{Peter Plavchan}{: GMU, Associate Professor, Undergraduate research advisor.}\\
\textbf{Joseph Weingartner}{: GMU,  Associate Professor, Undergraduate academic advisor.}\\

%-------------------------------------------------------------------------------
\end{resume}
\end{document}