%%%%%%%%%%%%%%%%%%%%%%%%%%%%%%%%%%%%%%%%%%%%%%%%%%%%%%%%%%%%%%%%%%%%%%%%%%%%%%%%
% Medium Length Graduate Curriculum Vitae
% LaTeX Template
% Version 1.2 (3/28/15)
%
% This template has been downloaded from:
% http://www.LaTeXTemplates.com
%
% Original author:
% Rensselaer Polytechnic Institute 
% (http://www.rpi.edu/dept/arc/training/latex/resumes/)
%
% Modified by:
% Daniel L Marks <xleafr@gmail.com> 3/28/2015
%
% Important note:
% This template requires the res.cls file to be in the same directory as the
% .tex file. The res.cls file provides the resume style used for structuring the
% document.
%
%%%%%%%%%%%%%%%%%%%%%%%%%%%%%%%%%%%%%%%%%%%%%%%%%%%%%%%%%%%%%%%%%%%%%%%%%%%%%%%%

%-------------------------------------------------------------------------------
%	PACKAGES AND OTHER DOCUMENT CONFIGURATIONS
%-------------------------------------------------------------------------------

%%%%%%%%%%%%%%%%%%%%%%%%%%%%%%%%%%%%%%%%%%%%%%%%%%%%%%%%%%%%%%%%%%%%%%%%%%%%%%%%
% You can have multiple style options the legal options ones are:
%
%   centered:	the name and address are centered at the top of the page 
%				(default)
%
%   line:		the name is the left with a horizontal line then the address to
%				the right
%
%   overlapped:	the section titles overlap the body text (default)
%
%   margin:		the section titles are to the left of the body text
%		
%   11pt:		use 11 point fonts instead of 10 point fonts
%
%   12pt:		use 12 point fonts instead of 10 point fonts
%
%%%%%%%%%%%%%%%%%%%%%%%%%%%%%%%%%%%%%%%%%%%%%%%%%%%%%%%%%%%%%%%%%%%%%%%%%%%%%%%%

\documentclass[marg, centered]{res}

% Default font is the helvetica postscript font
\usepackage{charter}
\usepackage{xcolor}
\usepackage{inputenc}
\usepackage[hidelinks]{hyperref}
\usepackage{csquotes}
\usepackage{orcidlink}
\usepackage{fontawesome}
\usepackage{fdsymbol}
\usepackage{multibib}
\usepackage{enumitem}
\usepackage[numbers]{natbib}
\usepackage[english]{babel}
\usepackage{etaremune,etoolbox}
\usepackage{fancyhdr}
\usepackage{lastpage}
\usepackage{geometry}
 \geometry{
 letterpaper,
 total={8.5in,11in},
 left=20mm,
 top=20mm,
 right=53.5mm,
 bottom=10mm,
 footskip=0mm
 }

\pagestyle{fancy}
\renewcommand{\headrulewidth}{0pt}
\fancyhf{}
\fancyhfoffset[l]{\dimexpr-20mm+53.5mm\relax}
\cfoot{Page \thepage \hspace{1pt} of \pageref{LastPage}}

\urlstyle{same}
\definecolor{dkbu}{HTML}{180f7d}

% define helper functions
\newcommand{\CVItem}[1]{
  \item\small{
    {#1 \vspace{-2pt}}
  }
}
\newcommand{\aap}{A\&A}
\newcommand{\mnras}{MNRAS}
\newcommand{\apjs}{ApJS}
\newcommand{\aj}{AJ}
\newcommand{\apj}{ApJ}
\newcommand{\apjl}{ApJL}
\newcommand{\pasp}{PASP}
\newcommand{\jatis}{JATIS}

% custom list
\newlist{talks}{itemize}{1}
\setlist[talks]{label=\textbf{?}}

% Increase text height
\textheight=700pt

\bibliographystyle{msacs}
\newcites{first}{{\bfseries First \& Second Author}\vspace{0.1cm}}
\bibliographystylefirst{msacs}
\newcites{co}{{\bfseries Coauthor}\vspace{0.1cm}}
\bibliographystyleco{msacs}

%%%------------------------------------------------------------------
%%% Reverse numbering of bibliography
\makeatletter
\AtBeginDocument{%%% natbib redefines the environment there
\renewenvironment{thebibliography}[1]
 {\bibsection\parindent\z@\bibpreamble\bibfont
  \settowidth{\dimen0}{#1.}%
  \setlength{\dimen2}{\dimen0}%
  \addtolength{\dimen2}{\labelsep}
  \begin{etaremune}[labelwidth=\dimen0,leftmargin=\dimen2]
  \ifNAT@openbib
    \renewcommand\newblock{\par}%
  \else
    \renewcommand\newblock{\hskip.11em \@plus .33em \@minus .07em}%
  \fi
  \sloppy
  \clubpenalty4000
  \widowpenalty4000
  \sfcode`\.\@m
  \let\NAT@bibitem@first@sw\@firstoftwo
  \let\citeN\cite\let\shortcite\cite\let\citeasnoun\cite}
 {\bibitem@fin\bibpostamble
  \def\@noitemerr{\PackageWarning{natbib}{Empty `thebibliography' environment}}%
  \end{etaremune}
  \bibcleanup}
}%%% end of \AtBeginDocument

%%% patch \@lbibitem to use only \item (for etaremune)
\patchcmd{\@lbibitem}{\item[\hfil\NAT@anchor{#2}{\NAT@num}]}{\item}{}{}
\makeatother
% Remove natbib titles that cause problems with asterisks on the left margin
\renewcommand{\bibsection}{}

\definecolor{orcidlogocol}{HTML}{A6CE39}

\begin{document}

%-------------------------------------------------------------------------------
%	NAME AND ADDRESS SECTION
%-------------------------------------------------------------------------------
\name{Michael A. Reefe}

% Note that addresses can be used for other contact information:
% -phone numbers
% -email addresses
% -linked-in profile

\address{  MIT Department of Physics
\\77 Massachusetts Avenue
\\Cambridge, MA 02139-4307}
\address{ \faMobile \hspace{0.2cm} +1 (540) 848-4434
\\ \faEnvelope \hspace{0.1cm} \href{mailto:mreefe@mit.edu}{\color{dkbu} mreefe@mit.edu}
\\ \faGlobe \hspace{0.15cm} \href{https://www.mit.edu/~mreefe/}{\color{dkbu}www.mit.edu/$\sim$mreefe/}}
\address{ \faGithub \hspace{0.1cm} \href{https://github.com/Michael-Reefe}{\color{dkbu} Michael-Reefe}
\\ \faLinkedinSquare \hspace{0.1cm} \href{https://www.linkedin.com/in/michael-reefe/}{\color{dkbu} Michael Reefe}
\\ \orcidlink{0000-0003-4701-8497} \hspace{0.1cm} \href{https://orcid.org/0000-0003-4701-8497}{\color{dkbu} 0000-0003-4701-8497}
}

\begin{resume}

% \section{{\scshape \bfseries Key Research Interests}}

% $\bullet$ Galaxy clusters, structure and dynamics of the intracluster medium, cool-core clusters \\
% $\bullet$ AGN feeding \& feedback, supermassive black hole \& host galaxy evolution \\
% $\bullet$ Multiwavelength astronomy, integral field spectroscopy, machine learning
% \vspace{-0.2cm}

%-------------------------------------------------------------------------------
%	EDUCATION SECTION
%-------------------------------------------------------------------------------
\section{{\scshape \bfseries Education}}
{\bf Ph.D. Candidate, Physics \hfill 2022 -- Present}\\
{Massachusetts Institute of Technology (MIT)} \hfill GPA: 5.0/5.0 \\
Advisor: {\em Prof. Michael McDonald} \\
\vspace{-0.7cm}

{\bf B.S., Physics $|$ Concentration in Astrophysics} $|$ {\em Summa cum laude} \hfill {\bf 2018 -- 2022} \\
{George Mason University (GMU)} \hfill GPA: 4.0/4.0 \\ 
Honors Thesis: {\em ``CLASS: Coronal Line Activity Spectroscopic Survey''} \\
Advisor: {\em Prof. Shobita Satyapal}

%-------------------------------------------------------------------------------

%-------------------------------------------------------------------------------
%	EXPERIENCE SECTION
%-------------------------------------------------------------------------------
% Modify the format of each position
\begin{format}
\title{l}\dates{r}\\
\body
\end{format}
%-------------------------------------------------------------------------------

\section{{\scshape \bfseries Research Experience}}

\dates{\textbf{2022 -- Present}}
\title{\textbf{NSF Graduate Research Fellow}, MIT}
\begin{position}
Advisor: \textit{Prof. Michael McDonald} \\
{\small Research Focuses: Galaxy clusters, structure and dynamics of the intracluster medium, cool-core clusters, AGN feeding and feedback, supermassive black hole and host galaxy evolution, multiwavelength astronomy, integral field spectroscopy.}
\end{position}

\vspace{-0.2cm}
\dates{\textbf{2021 -- 2022}}
\title{\textbf{Undergraduate Research Assistant}, GMU}
\begin{position}
Advisor: \textit{Prof. Shobita Satyapal} \\
{\small Research Focuses: Heavily obscured or dim/dwarf AGN, AGN feedback and host galaxy evolution, SDSS optical spectroscopy, integral field spectroscopy, coronal emission lines, cluster computing and parallelization, machine learning.} 
\end{position}

\vspace{-0.2cm}
\dates{\textbf{2019 -- 2021}}
\title{\textbf{Undergraduate Research Assistant}, GMU}
\begin{position}
Advisor: \textit{Prof. Peter Plavchan} \\
{\small Research Focuses: Exoplanet transits, radial velocities, fully automating the operations of GMU's 0.8 m telescope in Python, multi-band time-series photometry and spectroscopy.}
\end{position}

\section{{\scshape \bfseries Teaching Experience}}

{\textbf{Learning Assistant} $|$ PHYS 260: Electricity \& Magnetism, GMU \hfill \textbf{Fall 2019}} \\
{\small The undergraduate equivalent of a graduate teaching assistant position, including answering students' questions in class and holding office hours. Additionally included a final presentation on the challenges that the COVID-19 pandemic brought to this position, and how the other LAs and I managed them.}

%-------------------------------------------------------------------------------

\begin{format}
\title{l} \\
\body
\end{format}


\section{{\scshape \bfseries Honors \&\\ Awards}}

{Graduate Research Fellowship, NSF \hfill \textbf{2022 -- 2027}}
% \$37,000 stipend \& \$16,000 educational allowance per year for 3 years. Competitive national research fellowship for prospective graduate students across all science \& math disciplines that requires a detailed 3-year research proposal plan.
\vspace{-0.4cm}

{Whiteman Fellowship, MIT \hfill \textbf{2022 -- 2023}}
% MIT Physics Department fellowship covering the full stipend and tuition for the first year of study, funded by the Patrons of Physics Fellows at MIT.
\vspace{-0.4cm}

{Dean's Award for Excellence in Academics and Research, GMU \hfill \textbf{2022}}
% \$1,250 award.  GMU College of Science award for excellence in academics and/or research.
\vspace{-0.4cm}

{Outstanding Undergraduate Research Award, GMU \hfill \textbf{2022}}
% GMU Physics \& Astronomy department recognition of exceptional undergraduate research.
\vspace{-0.4cm}

{Outstanding Graduating Senior Award, GMU \hfill \textbf{2022}}
% GMU Physics \& Astronomy department recognition of an exceptional graduating senior.
\vspace{-0.4cm}

{Mason Distinction Scholarship, GMU \hfill \textbf{2018 -- 2022}}
% \$2,000 per year. Merit based scholarship for academically distinguished GMU students.
\vspace{-0.4cm}

{Outstanding Learning Assistant Award, GMU \hfill \textbf{2021}}
% \$150 award.  Recognition of outstanding leadership as a learning assistant.
\vspace{-0.4cm}

{Osher Lifelong Learning Institute Scholarship, GMU \hfill \textbf{2020}}
% \$500 award.  Recognition of academic excellence for GMU students.
\vspace{-0.0cm}

% {Dean's List, GMU \hfill \textbf{2018 -- 2022}}
% Cumulative GPA above 3.5 at GMU.
% \vspace{-0.0cm}

%-------------------------------------------------------------------------------
%	PROJECTS SECTION
%-------------------------------------------------------------------------------
% \newpage
\section{{\scshape \bfseries Refereed Publications}}

\vspace{-0.1cm}
\textbf{First Author}
\vspace{0.05cm}
\nocitefirst{*}
\bibliographyfirst{first}

\textbf{Coauthor}
\vspace{0.05cm}
\nociteco{*}
\bibliographyco{co}
\vspace{0.05cm}


\begin{format}
\title{l} \\
\employer{l} \\
\body
\end{format}

\section{{\scshape \bfseries Invited \& Contributed \\Talks} \\
\vspace{0.1cm}
{\mdseries\small $\smallblackdiamond$: Invited\\
$\smallblackcircle$: Contributed\\
$\smallblacksquare$: Poster}
% {\mdseries\small \faCommenting\,: Invited Talk\\ 
% \faComment\,: Contributed Talk\\
% \faFileText\,: Poster}
}

\textbf{Colloquia \& Seminars}

\begin{talks}[itemindent=0pt, leftmargin=19pt]

\item[{\small$\smallblackdiamond$}] MIT Kavli Institute for Astrophysics and Space Research $|$ Journal Club \hfill \textbf{Mar. 2023} \\
\textit{``On: AGN feedback in an infant galaxy cluster: the LOFAR-Chandra view of the giant FRII radio galaxy J103025$+$052430 at z$=$1.7''}

\item[\href{https://www.youtube.com/watch?v=xsyi9sqYH4o}{\color{dkbu}\small$\smallblackdiamond$}] GMU College of Science $|$ Graduation Ceremony \hfill \textbf{May 2022} \\
\textit{Invited to be the student speaker for the College of Science's graduation ceremony}.

\end{talks}

\textbf{Conferences}

\begin{talks}[itemindent=0pt, leftmargin=19pt]

\item[\href{https://cxc.cfa.harvard.edu/cdo/symposium_2024/schedule.html\#talk}{\color{dkbu}\small$\smallblackcircle$}] 25 Years of Science with Chandra $|$ Boston, MA \hfill \textbf{Dec. 2024} \\
\textit{``Mapping the Cooling Flow in the Phoenix Cluster with JWST and Chandra''} 

\item[\href{https://ui.adsabs.harvard.edu/abs/2024AAS...24344204R/abstract}{\color{dkbu}\small$\smallblackcircle$}] 243\textsuperscript{rd} Meeting of the American Astronomical Society $|$ New Orleans, LA \hfill \textbf{Jan. 2024} \\
\textit{``Shaken or stirred? Dynamics of the coronal temperature gas in the Phoenix Cluster''} 
    
\item[\href{https://ui.adsabs.harvard.edu/abs/2022AAS...24010113R/abstract}{\color{dkbu}\small$\smallblacksquare$}] 240\textsuperscript{th} Meeting of the American Astronomical Society $|$ Pasadena, CA \hfill \textbf{June 2022} \\
\textit{``A Large Scale Survey of Galaxies with Coronal Line Emission Selected from the Sloan Digital Sky Survey''}

\item[\href{https://zenodo.org/records/5114171}{\color{dkbu}\small$\smallblacksquare$}] TESS Science Conference II $|$ Virtual \hfill \textbf{Aug. 2021}\\
\textit{``A Flexible Python Observatory Automation Framework for the George Mason University Campus Telescope''}

\item[{\small$\smallblacksquare$}] GMU College of Science Undergraduate Research Colloquium $|$ Virtual \hfill \textbf{Apr. 2021}\\
\textit{``Automation of TESS Follow-up Observations with the GMU Campus Telescope''}

\item[\href{https://ui.adsabs.harvard.edu/abs/2021AAS...23734407R/abstract}{\color{dkbu}\small$\smallblacksquare$}] 237\textsuperscript{th} Meeting of the American Astronomical Society $|$ Virtual \hfill \textbf{Jan. 2021}\\
\textit{``An Asynchronous Object-Oriented Approach to Automation of the 0.8-meter George Mason University Campus Telescope in Python''}

\end{talks}


\section{{\scshape \bfseries Accepted Observing Proposals}}
\textbf{Co-I} \\
HST/COS Cycle 32 $|$ 24 orbits $|$ ID: 17716 \hfill \textbf{2024} \\
\textit{``Probing Multiphase Cooling Via OVI Emission in the Cores of the Most Extreme Cooling Flows''}

% Gemini Observatory: MAROON-X Instrument, 24$+$ nights requested \hfill 2021B \\
% Keck Observatory: HIRES Instrument, 5 nights requested \hfill 2021B \\
% NASA IRTF: iSHELL Instrument, 50 nights requested \hfill 2021B \\
% Gemini Observatory: MAROON-X Instrument, 24$+$ nights requested \hfill 2022A \\

\begin{format}
\title{l} \\
\body
\end{format}

\section{{\scshape \bfseries Service}}

\title{\textbf{V.P. of Academic Advocacy}, \href{https://physics-gsc.scripts.mit.edu/home/}{\color{dkbu} MIT Physics Graduate Student Council} \hfill \textbf{2024 -- Present}}
\begin{position}
{\small Served as the MIT Physics Graduate Student Council (PGSC)'s primary advocate for the students to the physics department leadership, holding regular meetings on how aspects of the PhD program and the department can be improved, i.e. more explicitly defined guidelines for academic advisors and academic advising meetings, uniform sets of expectations for the oral qualifying exam across the different physics divisions, implementing professional development requirements, etc.}
\end{position}

\vspace{-0.2cm}
\title{\textbf{Member}, \href{https://physics-gsc.scripts.mit.edu/home/gaga/}{\color{dkbu} MIT Physics Admissions Advisory Council} \hfill \textbf{2024 -- Present}}
\begin{position}
{\small Served on the Admissions Advisory Council, AKA Graduates Advising Graduate Admissions (GAGA), which is a subcommittee of the Physics Graduate Student Council that advises the MIT Chair of Graduate Admissions from the graduate student perspective and organizes the PhysGAAP program (see below).}
\end{position}

\vspace{-0.2cm}
\title{\textbf{Webmaster}, \href{https://physics-gsc.scripts.mit.edu/home/}{\color{dkbu} MIT Physics Graduate Student Council} \hfill \textbf{2023 -- Present}}
\begin{position}
{\small Served as the MIT PGSC's webmaster, maintaining the website, mailing lists, and calendar, and keeping them all up-to-date.}
\end{position}

\vspace{-0.2cm}
\title{\textbf{Mentor}, \href{https://sites.mit.edu/physgaap/}{\color{dkbu} MIT Physics Graduate Application Assistance Program} \hfill \textbf{Fall 2024}}
\begin{position}
{\small Served as a mentor in MIT's PhysGAAP Program, aiding prospective PhD students (primarily from under-represented groups) with the MIT Physics application, providing guidance on how to navigate the application and how best to present themselves.  I have mentored 3 prospective students through this program.}
\end{position}

\vspace{-0.2cm}
\title{\textbf{Organizer}, MKI Graduate Student Lunch \hfill \textbf{2023 -- 2024}}
\begin{position}
{\small Organized a weekly lunch and a talk series for the graduate students in the MIT Kavli Institute (MKI).}
\end{position}

\vspace{-0.2cm}
\title{\textbf{Faculty Search Undergraduate Liaison}, GMU Dept. of Physics \& Astronomy \hfill \textbf{Winter 2022}}
\begin{position}
{\small Worked as the undergraduate representative during a faculty search for a new astrophysics professor at GMU. Attended a mock lecture and research colloquium presented by each candidate, as well as interviews, and provided feedback to the faculty hiring committee from the undergraduate student perspective.}
\end{position}

\vspace{-0.2cm}
\title{\textbf{President}, GMU \href{https://gmuspectrum.squarespace.com/}{\color{dkbu} Spectrum} \hfill \textbf{2021 -- 2022}}
\begin{position}
{\small Planned talks, discussions, fundraisers, and other events, as well as managing website and budgetary concerns and working with the College of Science Faculty to improve diversity at GMU for student-led group Spectrum, which promotes the enhancement of under-represented groups in STEM.}
\end{position}

\vspace{-0.2cm}
\title{\textbf{Mentor}, GMU \href{https://gmuspectrum.squarespace.com/}{\color{dkbu} Spectrum} \hfill \textbf{2020 -- 2022}}
\begin{position}
{\small Provided academic and professional development tutoring for students in physics and astronomy at GMU through the student-led group Spectrum.}
\end{position}

\vspace{-0.2cm}
\title{\textbf{Panelist}, GMU Office of Fellowships NSF GRFP Cohort Workshop \hfill \textbf{July 2022}}
\begin{position}
{\small Served on a panel of NSF GRFP recipients and reviewers to answer students' questions about the application and review process.}
\end{position}

\vspace{-0.2cm}
\title{\textbf{Mentor}, \href{https://science.gmu.edu/assip}{\color{dkbu} Aspiring Scientists' Summer Internship Program} \hfill \textbf{Summer 2020, 2021}}
\begin{position}
{\small Taught high school interns about the research done in Prof. Peter Plavchan's group, and tutored them on how to perform it themselves to synthesize a presentable project by the end of the summer.}
\end{position}


\section{{\scshape \bfseries Outreach}}

\title{\textbf{Member}, \href{https://astrogazers.mit.edu/}{\color{dkbu} MIT Astrogazers} \hfill \textbf{2023 -- Present}}
\begin{position}
{\small As a member of the Astrogazers, I have been involved in engaging with the public at a number of sidewalk observing nights, exhibits at the annual Cambridge Science Festival, and other miscellaneous science-themed events in the greater Boston/Cambridge area.}
\end{position}

\vspace{-0.2cm}
\title{\textbf{Trivia Cohost}, \href{https://mitmuseum.mit.edu/programs/mit-museum-after-dark}{\color{dkbu} MIT Museum After Dark} \hfill \textbf{Dec. 2024}}
\begin{position}
{\small Worked with the Astrogazers to cohost trivia on the \textit{Hubble Space Telescope} at one of the MIT Museum's After Dark events themed around the 90s.}
\end{position}

\vspace{-0.2cm}
\title{\textbf{Volunteer}, \href{https://cambridgesciencefestival.org/}{\color{dkbu} Cambridge Science Festival} \hfill \textbf{Sep. 2023, 2024}}
\begin{position}
{\small Worked at the Astrogazers booths for the Cambridge Science Festival.}
\end{position}

\vspace{-0.2cm}
\title{\textbf{Panelist}, \href{https://science.gmu.edu/assip}{\color{dkbu} ASSIP} Career Day \hfill \textbf{Aug. 2022}}
\begin{position}
{\small Served on a panel of graduate students for a Career Day event hosted by GMU's ASSIP program, answering high school students' questions about a career in academia.}
\end{position}


%-------------------------------------------------------------------------------
%	COMPUTER SKILLS SECTION
%-------------------------------------------------------------------------------
\section{\scshape \bfseries Computer Skills}

\textbf{Coding}{: Python, Julia, MATLAB, Mathematica, Bash, Git} \\ 
\textbf{Python Packages}{: Numpy, Numba, Scipy, Astropy, Pandas, Matplotlib, Plotly} \\
\textbf{Astronomy Programs}{: DS9, AstroImageJ} \\
\textbf{Document Creation}{: \LaTeX, Vim, Microsoft Office} \\
%-------------------------------------------------------------------------------


%	Interests
%-------------------------------------------------------------------------------
% \newpage
\section{{\scshape \bfseries References}}
\textbf{Michael McDonald}{: MIT, Associate Professor, PhD research advisor.}\\
\textbf{Shobita Satyapal}{: GMU, Professor, Undergraduate research advisor.}\\
\textbf{Peter Plavchan}{: GMU, Associate Professor, Undergraduate research advisor.}\\
\textbf{Joseph Weingartner}{: GMU,  Associate Professor, Undergraduate academic advisor.}\\

%-------------------------------------------------------------------------------
\end{resume}
\end{document}